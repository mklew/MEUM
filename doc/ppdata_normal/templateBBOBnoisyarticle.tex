% This is based on "sig-alternate.tex" V2.0 May 2012
% This file should be compiled with V2.5 of "sig-alternate.cls" May 2012
%
% ----------------------------------------------------------------------------------------------------------------
%
% This .tex source is an example which *does* use
% the .bib file (from which the .bbl file % is produced).
% REMEMBER HOWEVER: After having produced the .bbl file,
% and prior to final submission, you *NEED* to 'insert'
% your .bbl file into your source .tex file so as to provide
% ONE 'self-contained' source file.
%
% Information on the sig-alternate class file and on the
% GECCO workshop paper format and submission can be found at these
% locations:
% http://www.acm.org/sigs/publications/proceedings-templates#aL2
% http://www.sheridanprinting.com/typedept/gecco3.htm
%
% ================= IF YOU HAVE QUESTIONS =======================
% Questions regarding the SIGS styles, SIGS policies and
% procedures, Conferences etc. should be sent to
% Adrienne Griscti (griscti@acm.org)
%
% Technical questions to bbob@lri.fr
% ===============================================================
%

\documentclass{sig-alternate}

\usepackage{graphicx}
\usepackage{rotating}
\usepackage{xcolor}  % color is sufficient
\pdfpagewidth=8.5in
\pdfpageheight=11in
\special{papersize=8.5in,11in}

    \renewcommand{\topfraction}{1}	% max fraction of floats at top
    \renewcommand{\bottomfraction}{1} % max fraction of floats at bottom
    %   Parameters for TEXT pages (not float pages):
    \setcounter{topnumber}{3}
    \setcounter{bottomnumber}{3}
    \setcounter{totalnumber}{3}     % 2 may work better
    \setcounter{dbltopnumber}{4}    % for 2-column pages
    \renewcommand{\dbltopfraction}{1}	% fit big float above 2-col. text
    \renewcommand{\textfraction}{0.0}	% allow minimal text w. figs
    %   Parameters for FLOAT pages (not text pages):
    \renewcommand{\floatpagefraction}{0.80}	% require fuller float pages
	% N.B.: floatpagefraction MUST be less than topfraction !!
    \renewcommand{\dblfloatpagefraction}{0.7}	% require fuller float pages

%%%%%%%%%%%%%%%%%%%%%%   END OF PREAMBLE   %%%%%%%%%%%%%%%%%%%%%%%%%%%%%%%%%%%%

%%%%%%%%%%%%%%%%%%%%%%%%%%%%%%%%%%%%%%%%%%%%%%%%%%%%%%%%%%%%%%%%%%%%%%%%%%%%%%%
%%%%%%%%% TO BE EDITED %%%%%%%%%%%%%%%%%%%%%%%%%%%%%%%%%%%%%%%%%%%%%%%%%%%%%%%%
%%%%%%%%%%%%%%%%%%%%%%%%%%%%%%%%%%%%%%%%%%%%%%%%%%%%%%%%%%%%%%%%%%%%%%%%%%%%%%%
% rungeneric.py writes data into a subfolder of ppdata
\newcommand{\bbobdatapath}{ppdata/} % default output folder of rungeneric.py
\input{\bbobdatapath bbob_pproc_commands.tex} % provide default of algname and algfolder
% \renewcommand{\algname}{MYNAME}  % name of algorithm as it should appear in the text
% \renewcommand{\algfolder}{ABC/} % subfolder of \\bbobdatapath for processed algorithm
%%%%%%%%%%%%%%%%%%%%%%%%%%%%%%%%%%%%%%%%%%%%%%%%%%%%%%%%%%%%%%%%%%%%%%%%%%%%%%%
%%%%%%%%%%%%%%%%%%%%%%%%%%%%%%%%%%%%%%%%%%%%%%%%%%%%%%%%%%%%%%%%%%%%%%%%%%%%%%%
%%%%%%%%%%%%%%%%%%%%%%%%%%%%%%%%%%%%%%%%%%%%%%%%%%%%%%%%%%%%%%%%%%%%%%%%%%%%%%%
\graphicspath{{\bbobdatapath\algfolder}}

\newcommand{\DIM}{\ensuremath{\mathrm{DIM}}}
\newcommand{\ERT}{\ensuremath{\mathrm{ERT}}}
\newcommand{\FEvals}{\ensuremath{\mathrm{FEvals}}}
\newcommand{\nruns}{\ensuremath{\mathrm{Nruns}}}
\newcommand{\Dfb}{\ensuremath{\Delta f_{\mathrm{best}}}}
\newcommand{\Df}{\ensuremath{\Delta f}}
\newcommand{\nbFEs}{\ensuremath{\mathrm{\#FEs}}}
\newcommand{\fopt}{\ensuremath{f_\mathrm{opt}}}
\newcommand{\ftarget}{\ensuremath{f_\mathrm{t}}}
\newcommand{\CrE}{\ensuremath{\mathrm{CrE}}}

\begin{document}
%
% --- Author Metadata here ---
\conferenceinfo{GECCO'13,} {July 6-10, 2013, Amsterdam, The Netherlands.}
\CopyrightYear{2013}
\crdata{TBA}
\clubpenalty=10000
\widowpenalty = 10000
% --- End of Author Metadata ---

\title{Black-Box Optimization Benchmarking Template for Noisy Function
Testbed
% \titlenote{If needed}
}
\subtitle{Draft version
\titlenote{Submission deadline: March 28th.}}
%Camera-ready paper due April 16th.}}

%
% You need the command \numberofauthors to handle the 'placement
% and alignment' of the authors beneath the title.
%
% For aesthetic reasons, we recommend 'three authors at a time'
% i.e. three 'name/affiliation blocks' be placed beneath the title.
%
% NOTE: You are NOT restricted in how many 'rows' of
% "name/affiliations" may appear. We just ask that you restrict
% the number of 'columns' to three.
%
% Because of the available 'opening page real-estate'
% we ask you to refrain from putting more than six authors
% (two rows with three columns) beneath the article title.
% More than six makes the first-page appear very cluttered indeed.
%
% Use the \alignauthor commands to handle the names
% and affiliations for an 'aesthetic maximum' of six authors.
% Add names, affiliations, addresses for
% the seventh etc. author(s) as the argument for the
% \additionalauthors command.
% These 'additional authors' will be output/set for you
% without further effort on your part as the last section in
% the body of your article BEFORE References or any Appendices.

\numberofauthors{1} %  in this sample file, there are a *total*
% of EIGHT authors. SIX appear on the 'first-page' (for formatting
% reasons) and the remaining two appear in the \additionalauthors section.
%
\author{
% You can go ahead and credit any number of authors here,
% e.g. one 'row of three' or two rows (consisting of one row of three
% and a second row of one, two or three).
%
% The command \alignauthor (no curly braces needed) should
% precede each author name, affiliation/snail-mail address and
% e-mail address. Additionally, tag each line of
% affiliation/address with \affaddr, and tag the
% e-mail address with \email.
%
% 1st. author
\alignauthor
Forename Name\\ %\titlenote{Dr.~Trovato insisted his name be first.}\\
%       \affaddr{Institute for Clarity in Documentation}\\
%       \affaddr{1932 Wallamaloo Lane}\\
%       \affaddr{Wallamaloo, New Zealand}\\
%       \email{trovato@corporation.com}
%% 2nd. author
%\alignauthor
%G.K.M. Tobin\titlenote{The secretary disavows
%any knowledge of this author's actions.}\\
%       \affaddr{Institute for Clarity in Documentation}\\
%       \affaddr{P.O. Box 1212}\\
%       \affaddr{Dublin, Ohio 43017-6221}\\
%       \email{webmaster@marysville-ohio.com}
%% 3rd. author
%\alignauthor Lars Th{\o}rv{\"a}ld\titlenote{This author is the
%one who did all the really hard work.}\\
%       \affaddr{The Th{\o}rv{\"a}ld Group}\\
%       \affaddr{1 Th{\o}rv{\"a}ld Circle}\\
%       \affaddr{Hekla, Iceland}\\
%       \email{larst@affiliation.org}
%\and  % use '\and' if you need 'another row' of author names
%% 4th. author
%\alignauthor Lawrence P. Leipuner\\
%       \affaddr{Brookhaven Laboratories}\\
%       \affaddr{Brookhaven National Lab}\\
%       \affaddr{P.O. Box 5000}\\
%       \email{lleipuner@researchlabs.org}
%% 5th. author
%\alignauthor Sean Fogarty\\
%       \affaddr{NASA Ames Research Center}\\
%       \affaddr{Moffett Field}\\
%       \affaddr{California 94035}\\
%       \email{fogartys@amesres.org}
%% 6th. author
%\alignauthor Charles Palmer\\
%       \affaddr{Palmer Research Laboratories}\\
%       \affaddr{8600 Datapoint Drive}\\
%       \affaddr{San Antonio, Texas 78229}\\
%       \email{cpalmer@prl.com}
} % author
%% There's nothing stopping you putting the seventh, eighth, etc.
%% author on the opening page (as the 'third row') but we ask,
%% for aesthetic reasons that you place these 'additional authors'
%% in the \additional authors block, viz.
%\additionalauthors{Additional authors: John Smith (The Th{\o}rv{\"a}ld Group,
%email: {\texttt{jsmith@affiliation.org}}) and Julius P.~Kumquat
%(The Kumquat Consortium, email: {\texttt{jpkumquat@consortium.net}}).}
%\date{30 July 1999}
%% Just remember to make sure that the TOTAL number of authors
%% is the number that will appear on the first page PLUS the
%% number that will appear in the \additionalauthors section.

\maketitle
\begin{abstract}
\end{abstract}

% Add any ACM category that you feel is needed
\category{G.1.6}{Numerical Analysis}{Optimization}[global optimization,
unconstrained optimization]
\category{F.2.1}{Analysis of Algorithms and Problem Complexity}{Numerical Algorithms and Problems}

% Complete with anything that is needed
\terms{Algorithms}

% Complete with anything that is needed
\keywords{Benchmarking, Black-box optimization}

% \section{Introduction}
%
% \section{Algorithm Presentation}
%
% \section{Experimental Procedure}
%
% \section{CPU Timing}
%
%%%%%%%%%%%%%%%%%%%%%%%%%%%%%%%%%%%%%%%%%%%%%%%%%%%%%%%%%%%%%%%%%%%%%%%%%%%%%%%
\section{Results}
%%%%%%%%%%%%%%%%%%%%%%%%%%%%%%%%%%%%%%%%%%%%%%%%%%%%%%%%%%%%%%%%%%%%%%%%%%%%%%%

Results of \algname\ from experiments according to \cite{hansen2012exp} on the benchmark
functions given in \cite{wp200902_2010,hansen2010noi} are presented in
Figures~\ref{fig:ERTgraphs}, \ref{fig:RLDs} and \ref{fig:ERTlogloss} and in
Tables~\ref{tab:ERTs} and \ref{tab:ERTloss}.
%%%%%%%%%%%%%%%%%%%%%%%%%%%%%%%%%%%%%%%%%%%%%%%%%%%%%%%%%%%%%%%%%%%%%%%%%%%%%%%
%%%%%%%%%%%%%%%%%%%%%%%%%%%%%%%%%%%%%%%%%%%%%%%%%%%%%%%%%%%%%%%%%%%%%%%%%%%%%%
%% version with one page of figures %%%%%%%%%%%%%%%%%%%%%%%%%%%%%%%%%%%%%%%%%
%%%%%%%%%%%%%%%%%%%%%%%%%%%%%%%%%%%%%%%%%%%%%%%%%%%%%%%%%%%%%%%%%%%%%%%%%%%%%%
\begin{figure*}
\begin{tabular}{l@{\hspace*{-0.021\textwidth}}l@{\hspace*{-0.021\textwidth}}l@{\hspace*{-0.021\textwidth}}l@{\hspace*{-0.021\textwidth}}l}
\hspace*{-0.021\textwidth}\includegraphics[trim=5mm 0mm 0mm 0mm, clip, width=0.22\textwidth,]{ppfigdim_f101}&
\includegraphics[trim=5mm 0mm 0mm 0mm, clip, width=0.22\textwidth]{ppfigdim_f104}&
\includegraphics[trim=5mm 0mm 0mm 0mm, clip, width=0.22\textwidth]{ppfigdim_f107}&
\includegraphics[trim=5mm 0mm 0mm 0mm, clip, width=0.22\textwidth]{ppfigdim_f110}&
\includegraphics[trim=5mm 0mm 0mm 0mm, clip, width=0.22\textwidth]{ppfigdim_f113}\\
\hspace*{-0.021\textwidth}\includegraphics[trim=5mm 0mm 0mm 0mm, clip, width=0.22\textwidth]{ppfigdim_f102}&
\includegraphics[trim=5mm 0mm 0mm 0mm, clip, width=0.22\textwidth]{ppfigdim_f105}&
\includegraphics[trim=5mm 0mm 0mm 0mm, clip, width=0.22\textwidth]{ppfigdim_f108}&
\includegraphics[trim=5mm 0mm 0mm 0mm, clip, width=0.22\textwidth]{ppfigdim_f111}&
\includegraphics[trim=5mm 0mm 0mm 0mm, clip, width=0.22\textwidth]{ppfigdim_f114}\\
\hspace*{-0.021\textwidth}\includegraphics[trim=5mm 0mm 0mm 0mm, clip, width=0.22\textwidth]{ppfigdim_f103}&
\includegraphics[trim=5mm 0mm 0mm 0mm, clip, width=0.22\textwidth]{ppfigdim_f106}&
\includegraphics[trim=5mm 0mm 0mm 0mm, clip, width=0.22\textwidth]{ppfigdim_f109}&
\includegraphics[trim=5mm 0mm 0mm 0mm, clip, width=0.22\textwidth]{ppfigdim_f112}&
\includegraphics[trim=5mm 0mm 0mm 0mm, clip, width=0.22\textwidth]{ppfigdim_f115}\\\hline
\hspace*{-0.021\textwidth}\includegraphics[trim=5mm 0mm 0mm 0mm, clip, width=0.22\textwidth]{ppfigdim_f116}&
\includegraphics[trim=5mm 0mm 0mm 0mm, clip, width=0.22\textwidth]{ppfigdim_f119}&
\includegraphics[trim=5mm 0mm 0mm 0mm, clip, width=0.22\textwidth]{ppfigdim_f122}&
\includegraphics[trim=5mm 0mm 0mm 0mm, clip, width=0.22\textwidth]{ppfigdim_f125}&
\includegraphics[trim=5mm 0mm 0mm 0mm, clip, width=0.22\textwidth]{ppfigdim_f128}\\
\hspace*{-0.021\textwidth}\includegraphics[trim=5mm 0mm 0mm 0mm, clip, width=0.22\textwidth]{ppfigdim_f117}&
\includegraphics[trim=5mm 0mm 0mm 0mm, clip, width=0.22\textwidth]{ppfigdim_f120}&
\includegraphics[trim=5mm 0mm 0mm 0mm, clip, width=0.22\textwidth]{ppfigdim_f123}&
\includegraphics[trim=5mm 0mm 0mm 0mm, clip, width=0.22\textwidth]{ppfigdim_f126}&
\includegraphics[trim=5mm 0mm 0mm 0mm, clip, width=0.22\textwidth]{ppfigdim_f129}\\
\hspace*{-0.021\textwidth}\includegraphics[trim=5mm 0mm 0mm 0mm, clip, width=0.22\textwidth]{ppfigdim_f118}&
\includegraphics[trim=5mm 0mm 0mm 0mm, clip, width=0.22\textwidth]{ppfigdim_f121}&
\includegraphics[trim=5mm 0mm 0mm 0mm, clip, width=0.22\textwidth]{ppfigdim_f124}&
\includegraphics[trim=5mm 0mm 0mm 0mm, clip, width=0.22\textwidth]{ppfigdim_f127}&
\includegraphics[trim=5mm 0mm 0mm 0mm, clip, width=0.22\textwidth]{ppfigdim_f130}\\
\end{tabular}
\vspace{-1ex}
 \caption{
 \label{fig:ERTgraphs}
 \bbobppfigdimlegend{$f_{101}$ and $f_{130}$}
 }
\end{figure*}
%%%%%%%%%%%%%%%%%%%%%%%%%%%%%%%%%%%%%%%%%%%%%%%%%%%%%%%%%%%%%%%%%%%%%%%%%%%%%%%
%\newcommand{\tablecaption}{
%Expected running time (ERT in number of function evaluations)
%divided by the best ERT measured during BBOB-2009 (given in the respective
%first row) for different $\Df$ values for functions
%$f_1$--$f_{24}$.
%The median number of conducted function evaluations is additionally given in 
%\textit{italics}, if $\ERT(10^{-7}) =\infty$.
%
%\#succ is the number of trials that reached the final target $\fopt + 10^{-8}$.
%}
%%%%%%%%%%%%%%%%%%%%%%%%%%%%%%%%%%%%%%%%%%%%%%%%%%%%%%%%%%%%%%%%%%%%%%%%%%%%%%%
\begin{table*}
% \centering
% \input{\bbobdatapath ppfigdim_f101}
% \input{\bbobdatapath ppfigdim_f102}
% \input{\bbobdatapath ppfigdim_f103}
% \input{\bbobdatapath ppfigdim_f104}
% \input{\bbobdatapath ppfigdim_f105}
% \input{\bbobdatapath ppfigdim_f106}
% \input{\bbobdatapath ppfigdim_f107}
% \input{\bbobdatapath ppfigdim_f108}
% \input{\bbobdatapath ppfigdim_f109}
% \input{\bbobdatapath ppfigdim_f110}
% \input{\bbobdatapath ppfigdim_f111}
% \input{\bbobdatapath ppfigdim_f112}
% \input{\bbobdatapath ppfigdim_f113}
% \input{\bbobdatapath ppfigdim_f114}
% \input{\bbobdatapath ppfigdim_f115}
% \input{\bbobdatapath ppfigdim_f116}
% \input{\bbobdatapath ppfigdim_f117}
% \input{\bbobdatapath ppfigdim_f118}
% \input{\bbobdatapath ppfigdim_f119}
% \input{\bbobdatapath ppfigdim_f120}\\
% \caption[Table of ERTs 1]{\label{tab:ERTs1}\tablecaption{$f_{101}$-$f_{120}$}
% }
% \end{table*}
% %%%%%%%%%%%%%%%%%%%%%%%%%%%%%%%%%%%%%%%%%%%%%%%%%%%%%%%%%%%%%%%%%%%%%%%%%%%%%%%

% %%%%%%%%%%%%%%%%%%%%%%%%%%%%%%%%%%%%%%%%%%%%%%%%%%%%%%%%%%%%%%%%%%%%%%%%%%%%%%%
% \begin{table*}
% \centering
% \input{\bbobdatapath ppfigdim_f121}
% \input{\bbobdatapath ppfigdim_f122}
% \input{\bbobdatapath ppfigdim_f123}
% \input{\bbobdatapath ppfigdim_f124}
% \input{\bbobdatapath ppfigdim_f125}
% \input{\bbobdatapath ppfigdim_f126}
% \input{\bbobdatapath ppfigdim_f127}
% \input{\bbobdatapath ppfigdim_f128}
% \input{\bbobdatapath ppfigdim_f129}
% \input{\bbobdatapath ppfigdim_f130}\\
\hfill5-D\hfill20-D\hfill~\\[1ex]
\centering \mbox{\tiny
\input{\bbobdatapath\algfolder pptable_05D_nzall}
\input{\bbobdatapath\algfolder pptable_20D_nzall}}\\
\caption[Table of ERTs]{\label{tab:ERTs}\bbobpptablecaption
%\caption[Table of ERTs 2]{\label{tab:ERTs2}\tablecaption{$f_{121}$-$f_{130}$}
}
\end{table*}
%%%%%%%%%%%%%%%%%%%%%%%%%%%%%%%%%%%%%%%%%%%%%%%%%%%%%%%%%%%%%%%%%%%%%%%%%%%%%%%

%%%%%%%%%%%%%%%%%%%%%%%%%%%%%%%%%%%%%%%%%%%%%%%%%%%%%%%%%%%%%%%%%%%%%%%%%%%%%%%
\newcommand{\rot}[2][2.5]{
  \hspace*{-3.5\baselineskip}%
  \begin{rotate}{90}\hspace{#1em}#2
  \end{rotate}}
\begin{figure*}
\begin{tabular}{l@{\hspace*{-0.025\textwidth}}l@{\hspace*{-0.00\textwidth}}|l@{\hspace*{-0.025\textwidth}}l}
\multicolumn{2}{c}{$D=5$} & \multicolumn{2}{c}{$D=20$}\\
\rot{moderate noise}
\includegraphics[width=0.268\textwidth]{pprldistr_05D_nzmod} & 
\includegraphics[width=0.2362\textwidth,trim=2.40cm 0 0 0, clip]{ppfvdistr_05D_nzmod} &
\includegraphics[width=0.268\textwidth]{pprldistr_20D_nzmod} &
\includegraphics[width=0.2362\textwidth,trim=2.40cm 0 0 0, clip]{ppfvdistr_20D_nzmod} \\
\rot{severe noise}
\includegraphics[width=0.268\textwidth]{pprldistr_05D_nzsev} &
\includegraphics[width=0.2362\textwidth,trim=2.40cm 0 0 0, clip]{ppfvdistr_05D_nzsev} &
\includegraphics[width=0.268\textwidth]{pprldistr_20D_nzsev} &
\includegraphics[width=0.2362\textwidth,trim=2.40cm 0 0 0, clip]{ppfvdistr_20D_nzsev} \\
\rot[0.5]{severe noise multimod.}
\includegraphics[width=0.268\textwidth]{pprldistr_05D_nzsmm} &
\includegraphics[width=0.2362\textwidth,trim=2.40cm 0 0 0, clip]{ppfvdistr_05D_nzsmm} &
\includegraphics[width=0.268\textwidth]{pprldistr_20D_nzsmm} &
\includegraphics[width=0.2362\textwidth,trim=2.40cm 0 0 0, clip]{ppfvdistr_20D_nzsmm}\\
\rot{all functions}
\includegraphics[width=0.268\textwidth]{pprldistr_05D_nzall} & 
\includegraphics[width=0.2362\textwidth,trim=2.40cm 0 0 0, clip]{ppfvdistr_05D_nzall} &
\includegraphics[width=0.268\textwidth]{pprldistr_20D_nzall} &
\includegraphics[width=0.2362\textwidth,trim=2.40cm 0 0 0, clip]{ppfvdistr_20D_nzall} 
\end{tabular}
\caption{\label{fig:RLDs}%
\bbobpprldistrlegend{}
%Empirical cumulative distribution functions (ECDFs), plotting the fraction of trials versus running time (left subplots) or versus \Df\ (right subplots).  The thick red line represents the best achieved results. Left subplots: ECDF of the running time (number of function evaluations), divided by
% search space dimension $D$, to fall below $\fopt+\Df$ with
% $\Df=10^{k}$, where $k$ is the first value in the legend. Right subplots: ECDF of the best achieved \Df\ divided by $10^k$ (upper left
% lines in continuation of the left subplot), and best achieved \Df\
% divided by $10^{-8}$ for running times of $D, 10\,D,
% 100\,D\dots$ function evaluations (from right
% to left cycling black-cyan-magenta).
% %Top row: all results from all functions; second row: moderate noise
% %functions; third row: severe noise functions; fourth row: severe noise
% %and highly-multimodal functions.
% The legends indicate the number of functions that were solved in at
% least one trial. FEvals denotes number of function evaluations,
% $D$ and \textsf{DIM} denote search space dimension, and \Df\ and \textsf{Df} denote the difference to the optimal function value.
% Light brown lines in the background show ECDFs for target value $10^{-8}$ of all algorithms benchmarked during BBOB-2009.}
}
\end{figure*}
%%%%%%%%%%%%%%%%%%%%%%%%%%%%%%%%%%%%%%%%%%%%%%%%%%%%%%%%%%%%%%%%%%%%%%%%%%%%%%%

%%%%%%%%%%%%%%%%%%%%%%%%%%%%%%%%%%%%%%%%%%%%%%%%%%%%%%%%%%%%%%%%%%%%%%%%%%%%%%%
%%%%%%%%%%%%%%%%%%%%%%%%%%%%%%%%%%%%%%%%%%%%%%%%%%%%%%%%%%%%%%%%%%%%%%%%%%%%%%%
\begin{figure}
\centering
\includegraphics[width=0.24\textwidth, trim=0 0 16mm 12mm, clip]{pplogloss_05D_nzall}%
\includegraphics[width=0.24\textwidth, trim=7mm 0 9mm 12mm, clip]{pplogloss_20D_nzall}
\\[-6.2ex]
\parbox{0.49\columnwidth}{\centering 5-D}%
\parbox{0.49\columnwidth}{\centering 20-D}\\[5ex]
%
\input{\bbobdatapath\algfolder pploglosstable_05D_nzall}\\
\input{\bbobdatapath\algfolder pploglosstable_20D_nzall}
\caption{\label{tab:ERTloss}%
\bbobloglosstablecaption{}
%\ERT\ loss ratio. The ERT of the considered algorithm, the budget, is shown in the first column. 
%For the loss ratio the budget is divided by the ERT for the respective best result from BBOB-2009 (see also Figure~\ref{fig:ERTlogloss}). The last row $\text{RL}_{\text{US}}/\text{D}$ gives the number of function evaluations in 
%unsuccessful runs divided by dimension. Shown are the smallest, 10\%-ile, 25\%-ile, 
%50\%-ile, 75\%-ile and 90\%-ile value (smaller values are better).
%The ERT Loss ratio equals to one for the respective best algorithm from BBOB-2009.
%Typical median values are between ten and hundred. 
}
\end{figure}
\begin{figure}
\begin{tabular}{@{}l@{}@{}l@{}}
\multicolumn{1}{c}{$D=5$} & \multicolumn{1}{c}{$D=20$}\\
%\rot{all functions}
%\hspace*{-2mm}
\rot{moderate noise}
\hspace*{-2mm}
\includegraphics[width=0.24\textwidth, trim=0 0 16mm 12mm, clip]{pplogloss_05D_nzmod} &
\includegraphics[width=0.24\textwidth, trim=7mm 0 9mm 12mm, clip]{pplogloss_20D_nzmod} \\[-2ex]
\rot{severe noise}
\hspace*{-2mm}
\includegraphics[width=0.24\textwidth, trim=0 0 16mm 12mm, clip]{pplogloss_05D_nzsev} &
\includegraphics[width=0.24\textwidth, trim=7mm 0 9mm 12mm, clip]{pplogloss_20D_nzsev} \\[-2ex]
\rot[0.5]{severe noise multimod.}
\hspace*{-2mm}
\includegraphics[width=0.24\textwidth, trim=0 0 16mm 12mm, clip]{pplogloss_05D_nzsmm} &
\includegraphics[width=0.24\textwidth, trim=7mm 0 9mm 12mm, clip]{pplogloss_20D_nzsmm}
\end{tabular}
\caption{\label{fig:ERTlogloss}%
\bbobloglossfigurecaption{}
%\ERT\ loss ratio vs.\ a given budget $\FEvals$. Each cross ({\color{blue}$+$}) represents a single function.
%%
%The target value \ftarget\ used for a given \FEvals\  
%%(see Figure~\ref{fig:ERTgraphs}) 
%is the smallest (best) recorded 
%function value such that $\ERT(\ftarget)\le\FEvals$ for the presented algorithm. 
%%
%Shown is \FEvals\ divided by the respective best $\ERT(\ftarget)$ from BBOB-2009 
%%
%for functions $f_{101}$--$f_{130}$ in 5-D and 20-D. 
%%
%% Each \ERT\ is multiplied by $\exp(\CrE)$ correcting for the parameter crafting effort. 
%Line: geometric mean. Box-Whisker error bar:
%25-75\%-ile with median (box), 10-90\%-ile
%(caps), and minimum and maximum \ERT\ loss ratio (points). The vertical line
%gives the maximal number of function evaluations in a single trial in this function subset.
}
\end{figure}
%%%%%%%%%%%%%%%%%%%%%%%%%%%%%%%%%%%%%%%%%%%%%%%%%%%%%%%%%%%%%%%%%%%%%%%%%%%%%%%

%
% The following two commands are all you need in the
% initial runs of your .tex file to
% produce the bibliography for the citations in your paper.
\bibliographystyle{abbrv}
\bibliography{bbob}  % bbob.bib is the name of the Bibliography in this case
% You must have a proper ".bib" file
%  and remember to run:
% latex bibtex latex latex
% to resolve all references
% to create the ~.bbl file.  Insert that ~.bbl file into
% the .tex source file and comment out
% the command \texttt{{\char'134}thebibliography}.
%
% ACM needs 'a single self-contained file'!
%

\clearpage % otherwise the last figure might be missing
\end{document}
