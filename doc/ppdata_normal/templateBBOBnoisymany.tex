% This is based on "sig-alternate.tex" V2.0 May 2012
% This file should be compiled with V2.5 of "sig-alternate.cls" May 2012
%
% ----------------------------------------------------------------------------------------------------------------
%
% This .tex source is an example which *does* use
% the .bib file (from which the .bbl file % is produced).
% REMEMBER HOWEVER: After having produced the .bbl file,
% and prior to final submission, you *NEED* to 'insert'
% your .bbl file into your source .tex file so as to provide
% ONE 'self-contained' source file.
%
% Information on the sig-alternate class file and on the
% GECCO workshop paper format and submission can be found at these
% locations:
% http://www.acm.org/sigs/publications/proceedings-templates#aL2
% http://www.sheridanprinting.com/typedept/gecco3.htm
%
% ================= IF YOU HAVE QUESTIONS =======================
% Questions regarding the SIGS styles, SIGS policies and
% procedures, Conferences etc. should be sent to
% Adrienne Griscti (griscti@acm.org)
%
% Technical questions to bbob@lri.fr
% ===============================================================
%

\documentclass{sig-alternate}
\pdfpagewidth=8.5in
\pdfpageheight=11in
\special{papersize=8.5in,11in}
    \renewcommand{\topfraction}{1}	% max fraction of floats at top
    \renewcommand{\bottomfraction}{1} % max fraction of floats at bottom
    %   Parameters for TEXT pages (not float pages):
    \setcounter{topnumber}{3}
    \setcounter{bottomnumber}{3}
    \setcounter{totalnumber}{3}     % 2 may work better
    \setcounter{dbltopnumber}{4}    % for 2-column pages
    \renewcommand{\dbltopfraction}{1}	% fit big float above 2-col. text
    \renewcommand{\textfraction}{0.0}	% allow minimal text w. figs
    %   Parameters for FLOAT pages (not text pages):
    \renewcommand{\floatpagefraction}{0.80}	% require fuller float pages
    % N.B.: floatpagefraction MUST be less than topfraction !!
    \renewcommand{\dblfloatpagefraction}{0.7}	% require fuller float pages

%%%%%%%%%%%%%%%%%%%%%%%%%%%%%%%%%%%%%%%%%%%%%%%%%%%%%%
% Packages
\usepackage{graphicx}
\usepackage{tabularx}
\usepackage{xcolor}
\usepackage{float}
\usepackage{rotating}
\usepackage{xstring} % for string operations
\usepackage{wasysym} % Table legend with symbols input from post-processing
\usepackage{MnSymbol} % Table legend with symbols input from post-processing

%%%%%%%%%%%%%%%%%%%%%%%%%%%%%%%%%%%%%%%%%%%%%%%%%%%%%%
% Definitions

% Algorithm names on the right of the ECDF figures can be modified by
% uncommenting the following lines and inputting some text in the last
% brackets, make sure the algorithms are in the same order as for the post-processing:
% \newcommand{\algaperfprof}{Algorithm A short name}
% \newcommand{\algbperfprof}{Algorithm B short name}
% \newcommand{\algcperfprof}{Algorithm C short name}
% \newcommand{\algdperfprof}{Algorithm C short name}
% ...
% Algorithm names as they appear in the tables, same thing
% \newcommand{\algatables}{\algaperfprof}  % first argument in the post-processing
% \newcommand{\algbtables}{\algbperfprof}  % first argument in the post-processing
% \newcommand{\algctables}{\algcperfprof}  % first argument in the post-processing
% \newcommand{\algdtables}{\algdperfprof}  % second argument in the post-processing
% ...
% location of pictures files
\newcommand{\bbobdatapath}{ppdata/}
\input{\bbobdatapath bbob_pproc_commands.tex}
\graphicspath{{\bbobdatapath}}

% pre-defined commands
\newcommand{\DIM}{\ensuremath{\mathrm{DIM}}}
\newcommand{\ERT}{\ensuremath{\mathrm{ERT}}}
\newcommand{\FEvals}{\ensuremath{\mathrm{FEvals}}}
\newcommand{\nruns}{\ensuremath{\mathrm{Nruns}}}
\newcommand{\Dfb}{\ensuremath{\Delta f_{\mathrm{best}}}}
\newcommand{\Df}{\ensuremath{\Delta f}}
\newcommand{\nbFEs}{\ensuremath{\mathrm{\#FEs}}}
\newcommand{\fopt}{\ensuremath{f_\mathrm{opt}}}
\newcommand{\ftarget}{\ensuremath{f_\mathrm{t}}}
\newcommand{\CrE}{\ensuremath{\mathrm{CrE}}}


%%%%%%%%%%%%%%%%%%%%%%%%%%%%%%%%%%%%%%%%%%%%%%%%%%%%%%

\begin{document}


%
% --- Author Metadata here ---
\conferenceinfo{GECCO'13,} {July 6-10, 2013, Amsterdam, The Netherlands.}
\CopyrightYear{2013}
\crdata{TBA}
\clubpenalty=10000
\widowpenalty = 10000
% --- End of Author Metadata ---

\title{Black-Box Optimization Benchmarking Template for the Comparison of More than Two Algorithms on the NoisyTestbed}
\subtitle{Draft version
\titlenote{Submission deadline: March 28th.}}
%Camera-ready paper due April 16th.}}

%
% You need the command \numberofauthors to handle the 'placement
% and alignment' of the authors beneath the title.
%
% For aesthetic reasons, we recommend 'three authors at a time'
% i.e. three 'name/affiliation blocks' be placed beneath the title.
%
% NOTE: You are NOT restricted in how many 'rows' of
% "name/affiliations" may appear. We just ask that you restrict
% the number of 'columns' to three.
%
% Because of the available 'opening page real-estate'
% we ask you to refrain from putting more than six authors
% (two rows with three columns) beneath the article title.
% More than six makes the first-page appear very cluttered indeed.
%
% Use the \alignauthor commands to handle the names
% and affiliations for an 'aesthetic maximum' of six authors.
% Add names, affiliations, addresses for
% the seventh etc. author(s) as the argument for the
% \additionalauthors command.
% These 'additional authors' will be output/set for you
% without further effort on your part as the last section in
% the body of your article BEFORE References or any Appendices.

\numberofauthors{1} %  in this sample file, there are a *total*
% of EIGHT authors. SIX appear on the 'first-page' (for formatting
% reasons) and the remaining two appear in the \additionalauthors section.
%
\author{
% You can go ahead and credit any number of authors here,
% e.g. one 'row of three' or two rows (consisting of one row of three
% and a second row of one, two or three).
%
% The command \alignauthor (no curly braces needed) should
% precede each author name, affiliation/snail-mail address and
% e-mail address. Additionally, tag each line of
% affiliation/address with \affaddr, and tag the
% e-mail address with \email.
%
% 1st. author
\alignauthor
Forename Name\\ %\titlenote{Dr.~Trovato insisted his name be first.}\\
%       \affaddr{Institute for Clarity in Documentation}\\
%       \affaddr{1932 Wallamaloo Lane}\\
%       \affaddr{Wallamaloo, New Zealand}\\
%       \email{trovato@corporation.com}
%% 2nd. author
%\alignauthor
%G.K.M. Tobin\titlenote{The secretary disavows
%any knowledge of this author's actions.}\\
%       \affaddr{Institute for Clarity in Documentation}\\
%       \affaddr{P.O. Box 1212}\\
%       \affaddr{Dublin, Ohio 43017-6221}\\
%       \email{webmaster@marysville-ohio.com}
%% 3rd. author
%\alignauthor Lars Th{\o}rv{\"a}ld\titlenote{This author is the
%one who did all the really hard work.}\\
%       \affaddr{The Th{\o}rv{\"a}ld Group}\\
%       \affaddr{1 Th{\o}rv{\"a}ld Circle}\\
%       \affaddr{Hekla, Iceland}\\
%       \email{larst@affiliation.org}
%\and  % use '\and' if you need 'another row' of author names
%% 4th. author
%\alignauthor Lawrence P. Leipuner\\
%       \affaddr{Brookhaven Laboratories}\\
%       \affaddr{Brookhaven National Lab}\\
%       \affaddr{P.O. Box 5000}\\
%       \email{lleipuner@researchlabs.org}
%% 5th. author
%\alignauthor Sean Fogarty\\
%       \affaddr{NASA Ames Research Center}\\
%       \affaddr{Moffett Field}\\
%       \affaddr{California 94035}\\
%       \email{fogartys@amesres.org}
%% 6th. author
%\alignauthor Charles Palmer\\
%       \affaddr{Palmer Research Laboratories}\\
%       \affaddr{8600 Datapoint Drive}\\
%       \affaddr{San Antonio, Texas 78229}\\
%       \email{cpalmer@prl.com}
} % author
%% There's nothing stopping you putting the seventh, eighth, etc.
%% author on the opening page (as the 'third row') but we ask,
%% for aesthetic reasons that you place these 'additional authors'
%% in the \additional authors block, viz.
%\additionalauthors{Additional authors: John Smith (The Th{\o}rv{\"a}ld Group,
%email: {\texttt{jsmith@affiliation.org}}) and Julius P.~Kumquat
%(The Kumquat Consortium, email: {\texttt{jpkumquat@consortium.net}}).}
%\date{30 July 1999}
%% Just remember to make sure that the TOTAL number of authors
%% is the number that will appear on the first page PLUS the
%% number that will appear in the \additionalauthors section.

\maketitle
\begin{abstract}
\end{abstract}

% Add any ACM category that you feel is needed
\category{G.1.6}{Numerical Analysis}{Optimization}[global optimization,
unconstrained optimization]
\category{F.2.1}{Analysis of Algorithms and Problem Complexity}{Numerical Algorithms and Problems}

% Complete with anything that is needed
\terms{Algorithms}

% Complete with anything that is needed
\keywords{Benchmarking, Black-box optimization}

% \section{Introduction}
%
% \section{Algorithm Presentation}
%
% \section{Experimental Procedure}
%
%%%%%%%%%%%%%%%%%%%%%%%%%%%%%%%%%%%%%%%%%%%%%%%%%%%%%%%%%%%%%%%%%%%%%%%%%%%%%%%
\section{Results}
%%%%%%%%%%%%%%%%%%%%%%%%%%%%%%%%%%%%%%%%%%%%%%%%%%%%%%%%%%%%%%%%%%%%%%%%%%%%%%%

Results from experiments according to \cite{hansen2012exp} on the benchmark
functions given in \cite{wp200902_2010,hansen2012noi} are presented in 
Figures~\ref{fig:ECDFs05D} and \ref{fig:ECDFs20D}, and Figure~\ref{fig:scaling}.
The \textbf{expected running time (\ERT)}, used in the figures and table,
depends on a given target function value, $\ftarget=\fopt+\Df$, and is computed
over all relevant trials as the number of function evaluations executed during
each trial while the best function value did not reach \ftarget, summed over
all trials and divided by the number of trials that actually reached \ftarget\
\cite{hansen2012exp,price1997dev}.
\textbf{Statistical significance} is tested with the rank-sum test for a given
target $\Delta\ftarget$ using, for each trial, either the number of needed
function evaluations to reach $\Delta\ftarget$ (inverted and multiplied by
$-1$), or, if the target was not reached, the best $\Df$-value achieved,
measured only up to the smallest number of overall function evaluations for any
unsuccessful trial under consideration if available.
Tables~\ref{tab:ERTs5} and~\ref{tab:ERTs20} give the Expected Running Time (\ERT) for targets
$10^{1,\,-1,\,-3,\,-5,\,-7}$ divided by the best \ERT\ obtained during
BBOB-2009 (given in the \ERT$_{\text{best}}$ row), respectively in 5-D
and 20-D.
Bold entries correspond to the best (or 3-best if there are more than 3
algorithms) values.
The median number of conducted function evaluations is additionally given in
\textit{italics}, if $\ERT(10^{-7}) =\infty$.
\#succ is the number of trials that reached the final target $\fopt + 10^{-8}$.
Entries with the $\downarrow$ symbol are statistically significantly better 
(according to the rank-sum test) compared to the best algorithm in BBOB-2009,
with $p=0.05$ or $p=10^{-k}$ where $k>1$ is the number
following the $\downarrow$ symbol, with Bonferroni correction of 30.

%%%%%%%%%%%%%%%%%%%%%%%%%%%%%%%%%%%%%%%%%%%%%%%%%%%%%%%%%%%%%%%%%%%%%%%%%%%%%%%
\begin{figure*}
\centering
\begin{tabular}{@{}c@{}c@{}c@{}c@{}c@{}}
\includegraphics[width=0.2\textwidth, trim=20mm 7mm 15mm 3mm, clip]{ppfigs_f101}&
\includegraphics[width=0.2\textwidth, trim=20mm 7mm 15mm 3mm, clip]{ppfigs_f104}&
\includegraphics[width=0.2\textwidth, trim=20mm 7mm 15mm 3mm, clip]{ppfigs_f107}&
\includegraphics[width=0.2\textwidth, trim=20mm 7mm 15mm 3mm, clip]{ppfigs_f110}&
\includegraphics[width=0.2\textwidth, trim=20mm 7mm 15mm 3mm, clip]{ppfigs_f113}\\
\includegraphics[width=0.2\textwidth, trim=20mm 7mm 15mm 3mm, clip]{ppfigs_f102}&
\includegraphics[width=0.2\textwidth, trim=20mm 7mm 15mm 3mm, clip]{ppfigs_f105}&
\includegraphics[width=0.2\textwidth, trim=20mm 7mm 15mm 3mm, clip]{ppfigs_f108}&
\includegraphics[width=0.2\textwidth, trim=20mm 7mm 15mm 3mm, clip]{ppfigs_f111}&
\includegraphics[width=0.2\textwidth, trim=20mm 7mm 15mm 3mm, clip]{ppfigs_f114}\\
\includegraphics[width=0.2\textwidth, trim=20mm 7mm 15mm 3mm, clip]{ppfigs_f103}&
\includegraphics[width=0.2\textwidth, trim=20mm 7mm 15mm 3mm, clip]{ppfigs_f106}&
\includegraphics[width=0.2\textwidth, trim=20mm 7mm 15mm 3mm, clip]{ppfigs_f109}&
\includegraphics[width=0.2\textwidth, trim=20mm 7mm 15mm 3mm, clip]{ppfigs_f112}&
\includegraphics[width=0.2\textwidth, trim=20mm 7mm 15mm 3mm, clip]{ppfigs_f115}\\
\includegraphics[width=0.2\textwidth, trim=20mm 7mm 15mm 3mm, clip]{ppfigs_f116}&
\includegraphics[width=0.2\textwidth, trim=20mm 7mm 15mm 3mm, clip]{ppfigs_f119}&
\includegraphics[width=0.2\textwidth, trim=20mm 7mm 15mm 3mm, clip]{ppfigs_f122}&
\includegraphics[width=0.2\textwidth, trim=20mm 7mm 15mm 3mm, clip]{ppfigs_f125}&
\includegraphics[width=0.2\textwidth, trim=20mm 7mm 15mm 3mm, clip]{ppfigs_f128}\\
\includegraphics[width=0.2\textwidth, trim=20mm 7mm 15mm 3mm, clip]{ppfigs_f117}&
\includegraphics[width=0.2\textwidth, trim=20mm 7mm 15mm 3mm, clip]{ppfigs_f120}&
\includegraphics[width=0.2\textwidth, trim=20mm 7mm 15mm 3mm, clip]{ppfigs_f123}&
\includegraphics[width=0.2\textwidth, trim=20mm 7mm 15mm 3mm, clip]{ppfigs_f126}&
\includegraphics[width=0.2\textwidth, trim=20mm 7mm 15mm 3mm, clip]{ppfigs_f129}\\
\includegraphics[width=0.2\textwidth, trim=20mm 7mm 15mm 3mm, clip]{ppfigs_f118}&
\includegraphics[width=0.2\textwidth, trim=20mm 7mm 15mm 3mm, clip]{ppfigs_f121}&
\includegraphics[width=0.2\textwidth, trim=20mm 7mm 15mm 3mm, clip]{ppfigs_f124}&
\includegraphics[width=0.2\textwidth, trim=20mm 7mm 15mm 3mm, clip]{ppfigs_f127}&
\includegraphics[width=0.2\textwidth, trim=20mm 7mm 15mm 3mm, clip]{ppfigs_f130}
\end{tabular}
\caption[Expected running time (\ERT) divided by dimension
versus dimension in log-log presentation]{\label{fig:scaling}%
\bbobppfigslegend{$f_{101}$ and $f_{130}$}
%Expected running
%time (\ERT) divided by dimension  for
%target function value $10^{-8}$ as $\log_{10}$ values versus dimension. Different% symbols
%correspond to different algorithms given in legend of $f_{101}$ and $f_{130}$.
%Light symbols give the maximum number of function evaluations from all trials div%ided by the
%dimension. Horizontal lines give linear scaling,
%the slanted dotted lines give quadratic scaling.
%\input{\bbobdatapath ppfigs}
}
\end{figure*}
%%%%%%%%%%%%%%%%%%%%%%%%%%%%%%%%%%%%%%%%%%%%%%%%%%%%%%%%%%%%%%%%%%%%%%%%%%%%%%%
%%%%%%%%%%%%%%%%%%%%%%%%%%%%%%%%%%%%%%%%%%%%%%%%%%%%%%%%%%%%%%%%%%%%%%%%%%%%%%%
\newcommand{\rot}[2][2.5]{
  \hspace*{-3.5\baselineskip}%
  \begin{rotate}{90}\hspace{#1em}#2
  \end{rotate}}
\newcommand{\includeperfprof}[1]{% include and annotate at the side
  \input{\bbobdatapath #1}%
  \includegraphics[width=0.4135\textwidth,trim=0mm 0mm 34mm 10mm, clip]{#1}%
  \raisebox{.037\textwidth}{\parbox[b][.3\textwidth]{.0868\textwidth}{\begin{scriptsize}
    \perfprofsidepanel % this is "\algaperfprof \vfill \algbperfprof \vfill" etc
  \end{scriptsize}}}
}
%%%%%%%%%%%%%%%%%%%%%%%%%%%%%%%%%%%%%%%%%%%%%%%%%%%%%%%%%%%%%%%%%%%%%%%%%%%%%%%
%%%%%%%%%%%%%%%%%%%%%%%%%%%%%%%%%%%%%%%%%%%%%%%%%%%%%%%%%%%%%%%%%%%%%%%%%%%%%%%
\begin{figure*}
 \begin{tabular}{@{}c@{}c@{}}
all functions & moderate noise \\
 \includeperfprof{pprldmany_05D_nzall} & 
 \includeperfprof{pprldmany_05D_nzmod} \\
severe noise & severe noise multimod.\\
 \includeperfprof{pprldmany_05D_nzsev} & 
 \includeperfprof{pprldmany_05D_nzsmm}
 \end{tabular}
\caption{
\label{fig:ECDFs05D}
Bootstrapped empirical cumulative distribution of 
the number of objective function evaluations
% \# of $f$-evaluations 
divided by dimension (FEvals/D) for 50 targets in
$10^{[-8..2]}$ for all functions and subgroups in 5-D. The ``best 2009'' line
corresponds to the best \ERT\ observed during BBOB 2009 for each single target. 
}
\end{figure*}
%%%%%%%%%%%%%%%%%%%%%%%%%%%%%%%%%%%%%%%%%%%%%%%%%%%%%%%%%%%%%%%%%%%%%%%%%%%%%%%
%%%%%%%%%%%%%%%%%%%%%%%%%%%%%%%%%%%%%%%%%%%%%%%%%%%%%%%%%%%%%%%%%%%%%%%%%%%%%%%
\begin{figure*}
 \begin{tabular}{@{}c@{}c@{}}
all functions & moderate noise \\
 \includeperfprof{pprldmany_20D_nzall} & 
 \includeperfprof{pprldmany_20D_nzmod} \\
severe noise & severe noise multimod.\\
 \includeperfprof{pprldmany_20D_nzsev} & 
 \includeperfprof{pprldmany_20D_nzsmm}
 \end{tabular}
\caption{
\label{fig:ECDFs20D}
Bootstrapped empirical cumulative distribution of 
the number of objective function evaluations
% \# of $f$-evaluations 
divided by dimension (FEvals/D) for 50 targets in
$10^{[-8..2]}$ for all functions and subgroups in 20-D. The ``best 2009'' line
corresponds to the best \ERT\ observed during BBOB 2009 for each single target. 
}
\end{figure*}
% \end{document}
%%%%%%%%%%%%%%%%%%%%%%%%%%%%%%%%%%%%%%%%%%%%%%%%%%%%%%%%%%%%%%%%%%%%%%%%%%%%%%%
%%%%%%%%%%%%%%%%%%%%%%%%%%%%%%%%%%%%%%%%%%%%%%%%%%%%%%%%%%%%%%%%%%%%%%%%%%%%%%%
\begin{table*}\tiny
%\hfill5-D\hfill~\\[1ex]
\mbox{\begin{minipage}[t]{0.49\textwidth}\tiny
\centering
\input{\bbobdatapath pptables_f101_05D} 

\input{\bbobdatapath pptables_f102_05D}

\input{\bbobdatapath pptables_f103_05D}

\input{\bbobdatapath pptables_f104_05D}

\input{\bbobdatapath pptables_f105_05D}

\input{\bbobdatapath pptables_f106_05D}

\input{\bbobdatapath pptables_f107_05D}

\input{\bbobdatapath pptables_f108_05D}

\input{\bbobdatapath pptables_f109_05D}

\input{\bbobdatapath pptables_f110_05D}

\input{\bbobdatapath pptables_f111_05D}

\input{\bbobdatapath pptables_f112_05D}

\input{\bbobdatapath pptables_f113_05D}

\input{\bbobdatapath pptables_f114_05D}

\input{\bbobdatapath pptables_f115_05D}
\end{minipage}
\begin{minipage}[t]{0.49\textwidth}\tiny
\centering

\input{\bbobdatapath pptables_f116_05D}

\input{\bbobdatapath pptables_f117_05D}

\input{\bbobdatapath pptables_f118_05D}

\input{\bbobdatapath pptables_f119_05D}

\input{\bbobdatapath pptables_f120_05D}

\input{\bbobdatapath pptables_f121_05D}

\input{\bbobdatapath pptables_f122_05D}

\input{\bbobdatapath pptables_f123_05D}

\input{\bbobdatapath pptables_f124_05D}

\input{\bbobdatapath pptables_f125_05D}

\input{\bbobdatapath pptables_f126_05D}

\input{\bbobdatapath pptables_f127_05D}

\input{\bbobdatapath pptables_f128_05D}

\input{\bbobdatapath pptables_f129_05D}

\input{\bbobdatapath pptables_f130_05D}
\end{minipage}}

 \caption{\label{tab:ERTs5}
\bbobpptablesmanylegend{dimension $5$}
% Expected running time (ERT in number of function evaluations)
% divided by the best ERT measured during BBOB-2009 (given in the respective
% first row) for different $\Df$ values for functions
% $f_{101}$--$f_{130}$ for dimension 5.
% The median number of conducted function evaluations is additionally given in 
% \textit{italics}, if $\ERT(10^{-7}) =\infty$.
% %
% \#succ is the number of trials that reached the final target $\fopt + 10^{-8}$.
 }
% 0:\:\algorithmAshort\ is \algorithmA\ and 1:\:\algorithmBshort\ is \algorithmB. 
% %
% Bold entries are statistically significantly better compared 
% to the other algorithm, with $p=0.05$ or $p=10^{-k}$ where $k>1$ is the number 
% following the $\star$ symbol, with Bonferroni
% correction of 48. 
% }
% command definition of \algxxxshort ends here
\end{table*}


\begin{table*}\tiny
%\hfill20-D\hfill~\\[1ex]
\mbox{\begin{minipage}[t]{0.49\textwidth}\tiny
\centering
\input{\bbobdatapath pptables_f101_20D} 

\input{\bbobdatapath pptables_f102_20D}

\input{\bbobdatapath pptables_f103_20D}

\input{\bbobdatapath pptables_f104_20D}

\input{\bbobdatapath pptables_f105_20D}

\input{\bbobdatapath pptables_f106_20D}

\input{\bbobdatapath pptables_f107_20D}

\input{\bbobdatapath pptables_f108_20D}

\input{\bbobdatapath pptables_f109_20D}

\input{\bbobdatapath pptables_f110_20D}

\input{\bbobdatapath pptables_f111_20D}

\input{\bbobdatapath pptables_f112_20D}

\input{\bbobdatapath pptables_f113_20D}

\input{\bbobdatapath pptables_f114_20D}

\input{\bbobdatapath pptables_f115_20D}
\end{minipage}
\begin{minipage}[t]{0.49\textwidth}\tiny
\centering

\input{\bbobdatapath pptables_f116_20D}

\input{\bbobdatapath pptables_f117_20D}

\input{\bbobdatapath pptables_f118_20D}

\input{\bbobdatapath pptables_f119_20D}

\input{\bbobdatapath pptables_f120_20D}

\input{\bbobdatapath pptables_f121_20D}

\input{\bbobdatapath pptables_f122_20D}

\input{\bbobdatapath pptables_f123_20D}

\input{\bbobdatapath pptables_f124_20D}

\input{\bbobdatapath pptables_f125_20D}

\input{\bbobdatapath pptables_f126_20D}

\input{\bbobdatapath pptables_f127_20D}

\input{\bbobdatapath pptables_f128_20D}

\input{\bbobdatapath pptables_f129_20D}

\input{\bbobdatapath pptables_f130_20D}
\end{minipage}}
 \caption{\label{tab:ERTs20}
 \bbobpptablesmanylegend{dimension $20$}
% Expected running time (ERT in number of function evaluations)
% divided by the best ERT measured during BBOB-2009 (given in the respective
% first row) for different $\Df$ values for functions
% $f_{101}$--$f_{130}$ for dimension 20.
% The median number of conducted function evaluations is additionally given in 
% \textit{italics}, if $\ERT(10^{-7}) =\infty$.
% %
% \#succ is the number of trials that reached the final target $\fopt + 10^{-8}$.
}
% 0:\:\algorithmAshort\ is \algorithmA\ and 1:\:\algorithmBshort\ is \algorithmB. 
% %
% Bold entries are statistically significantly better compared 
% to the other algorithm, with $p=0.05$ or $p=10^{-k}$ where $k>1$ is the number 
% following the $\star$ symbol, with Bonferroni
% correction of 48. 
% }
% command definition of \algxxxshort ends here
\end{table*}

%%%%%%%%%%%%%%%%%%%%%%%%%%%%%%%%%%%%%%%%%%%%%%%%%%%%%%%%%%%%%%%%%%%%%%%%%%%%%%%
%%%%%%%%%%%%%%%%%%%%%%%%%%%%%%%%%%%%%%%%%%%%%%%%%%%%%%%%%%%%%%%%%%%%%%%%%%%%%%%
%
% The following two commands are all you need in the
% initial runs of your .tex file to
% produce the bibliography for the citations in your paper.
\bibliographystyle{abbrv}
\bibliography{bbob}  % bbob.bib is the name of the Bibliography in this case
% You must have a proper ".bib" file
%  and remember to run:
% latex bibtex latex latex
% to resolve all references
% to create the ~.bbl file.  Insert that ~.bbl file into
% the .tex source file and comment out
% the command \texttt{{\char'134}thebibliography}.
%
% ACM needs 'a single self-contained file'!
%

% \clearpage % otherwise the last figure might be missing
\end{document}

%%%%%%%%%%%%%%%%%%%%%%%%%%%%%%%%%%%%%%%%%%%%%%%%%%%%%%%%%%%%%%%%%%%%%%%%%%%%%%%%%%%%%%%%%%%
